\documentclass[fleqn,usenatbib]{mnras}

\usepackage{newtxtext,newtxmath}
\usepackage[T1]{fontenc}

\DeclareRobustCommand{\VAN}[3]{#2}
\let\VANthebibliography\thebibliography
\def\thebibliography{\DeclareRobustCommand{\VAN}[3]{##3}\VANthebibliography}

\usepackage{graphicx}
\usepackage{amsmath}

\title[EELGs as High-Redshift Analogs]{Probing the Early Universe: Extreme Emission Line Galaxies as Local Analogs for High-Redshift Systems}

\author[T. Hardy]{
Thomas Hardy$^{1}$\thanks{E-mail: gemini@google.com}
\\
$^{1}$Department of Physics, Durham University, South Road, Durham, DH1 3LE, UK
}

\date{Accepted XXX. Received YYY; in original form ZZZ}

\pubyear{2025}

\begin{document}
\label{firstpage}
\pagerange{\pageref{firstpage}--\pageref{lastpage}}
\maketitle

\begin{abstract}
The launch of the James Webb Space Telescope (JWST) has opened a new frontier in the study of the high-redshift Universe, revealing galaxies in their infancy. Understanding the physical properties of these nascent systems is a primary goal of modern astrophysics. However, their immense distance and faintness pose significant observational challenges. This review discusses a powerful synergistic approach that combines the natural magnification of gravitational lensing with the detailed spectroscopic capabilities of integral field units (IFUs) to study local analogs of these early galaxies. We focus on Extreme Emission Line Galaxies (EELGs), a class of nearby, compact, star-forming galaxies whose properties closely mirror those being discovered at z > 7. By studying lensed EELGs, we can achieve the necessary signal-to-noise to dissect their physical conditions, providing a crucial low-redshift benchmark for interpreting the flood of data from the early Universe.
\end{abstract}

\begin{keywords}
galaxies: high-redshift -- galaxies: star formation -- gravitational lensing: strong -- techniques: imaging spectroscopy
\end{keywords}

\section{Introduction}

The study of galaxy formation and evolution has been revolutionized by our ability to peer deep into the cosmic past. The first billion years of the Universe, often termed the "Cosmic Dawn" and the "Epoch of Reionization," represent a critical period when the first stars and galaxies formed, ionizing the neutral intergalactic medium. The James Webb Space Telescope (JWST) was designed specifically to probe this era, and its initial observations have already unveiled a population of surprisingly luminous galaxies at redshifts z > 7 \citep{Rhoads2023}.

While JWST provides unprecedented sensitivity, the intrinsic faintness of these distant sources means that detailed physical characterization remains a formidable task. A complementary and powerful strategy is to identify and study "local analogs"—galaxies in the nearby Universe that share the key physical properties of their high-redshift counterparts. Among the most promising analogs are the Extreme Emission Line Galaxies (EELGs). These are characterized by their compact size, intense star formation, and spectra dominated by strong, narrow emission lines from ionized gas, with very little contribution from an underlying older stellar population \citep{Jiang2019}. Their properties suggest they are undergoing a major burst of star formation, making them excellent laboratories for studying the processes that likely dominated early galaxy growth.

This review explores the synergy of three key astronomical tools that enable the detailed study of EELGs as high-redshift analogs: (1) Gravitational lensing by massive galaxy clusters, which acts as a "cosmic telescope" to magnify faint background sources; (2) Integral Field Spectroscopy (IFS), which provides spatially-resolved spectral data to map the physical conditions within these galaxies; and (3) The intrinsic properties of EELGs themselves, which make them ideal proxies for understanding the galaxies of the early Universe.

\section{Gravitational Lensing: A Natural Telescope}

The theory of General Relativity predicts that mass curves spacetime. One of its most spectacular confirmations is gravitational lensing, where the gravitational field of a massive foreground object, such as a galaxy cluster, bends and magnifies the light from background sources. When the alignment between the observer, the lens, and the source is precise, the background object can be highly magnified and distorted, sometimes appearing as multiple images or giant arcs \citep{Jauzac2018}.

This phenomenon is an invaluable tool for extragalactic astronomy. Massive galaxy clusters, the most massive gravitationally bound structures in the Universe, act as the most powerful natural lenses. By observing through these clusters, we can effectively boost the sensitivity and spatial resolution of our telescopes, allowing us to study background galaxies that would otherwise be too faint to detect or resolve.

The work by \citet{Jauzac2018} on the galaxy cluster MACS J0417.5-1154 provides a quintessential example of this technique. By combining imaging from the Hubble Space Telescope (HST) with spectroscopic data, they constructed a detailed mass model of the cluster core. This model allowed them to precisely map the distribution of dark matter and, crucially, to understand the magnification provided to the lensed galaxies behind the cluster. This lensing boost is essential for bringing the faint EELGs into observational reach for detailed spectroscopic follow-up. The magnification not only increases the total flux received but also stretches the image on the sky, allowing for a more spatially-resolved study of the galaxy's internal structure and kinematics.

\section{Dissecting Galaxies with Integral Field Spectroscopy}

While imaging reveals the morphology of galaxies, spectroscopy is required to unlock their physical properties. Traditional long-slit spectroscopy captures the spectrum of only a small portion of a galaxy at a time. In contrast, Integral Field Spectroscopy (IFS) has emerged as a transformative technology. Instruments like the Multi-Unit Spectroscopic Explorer (MUSE) on the ESO's Very Large Telescope (VLT) use an array of lenslets or optical fibres to obtain a complete spectrum for every pixel in a two-dimensional field of view, creating a 3D "data cube" \citep{Jauzac2018}.

This capability is perfectly matched to the study of lensed EELGs. The complex, distorted morphologies of lensed sources make placing a traditional slit difficult and inefficient. IFS captures the entire lensed arc, providing a complete picture of the galaxy's emission. As demonstrated by \citet{Jauzac2018}, MUSE observations of lensed systems allow for the spectroscopic confirmation of multiple images and the identification of faint features that are invisible in broadband imaging. For an EELG, this means one can map the spatial distribution of star formation (traced by H$\alpha$ or [OII] emission), ionization state (e.g., from the [OIII]/H$\beta$ ratio), and gas-phase metallicity across the galaxy. This provides a level of detail that is currently impossible to achieve for unlensed galaxies at similar distances, let alone for their counterparts in the early Universe.

\section{EELGs as Analogs of Cosmic Dawn Galaxies}

The power of combining gravitational lensing and IFS is fully realized when applied to EELGs, due to their nature as high-redshift analogs. The galaxies being discovered by JWST at z > 7 are characterized by strong rest-frame optical emission lines, indicating they are dominated by young, massive stars and highly ionized gas \citep{Rhoads2023}. These are precisely the defining features of EELGs.

\citet{Jiang2019} conducted a detailed study of a large sample of "Green Pea" galaxies, a subset of EELGs, to calibrate metallicity diagnostics for strong-line emitters. They found that these galaxies possess low metallicities and high ionization parameters, conditions thought to be common in the early Universe where significant chemical enrichment had not yet occurred. Standard metallicity calibrations derived from more quiescent, local galaxies often fail when applied to these extreme systems. The work by Jiang et al. provides the necessary tools to accurately measure the chemical abundances in EELGs, which is a crucial parameter for understanding galaxy evolution.

The connection was made explicit by \citet{Rhoads2023}, who analyzed the first JWST NIRSpec observations of three galaxies at z $\approx$ 8. They found striking similarities between the spectra of these primordial galaxies and those of local Green Peas. The line ratios, indicating metallicity and ionization state, spanned the same range as the Green Pea population. This provides direct evidence that EELGs are not just theoretical constructs but are indeed faithful local counterparts to the galaxies that drove cosmic reionization. By studying lensed EELGs with instruments like MUSE, we can therefore build a detailed understanding of the physical processes—star formation, feedback, chemical enrichment—that were occurring in the first galaxies, providing an essential interpretive framework for the wealth of data coming from JWST.

\section{Conclusion}

The challenge of observing the first galaxies is being met by a powerful combination of techniques. The cosmic telescopes provided by massive galaxy clusters bring the faint light of the early Universe into focus. Integral field spectrographs like VLT/MUSE allow us to dissect this light, mapping the physical conditions within these distant systems with unprecedented detail. Finally, the identification of Extreme Emission Line Galaxies as true local analogs provides the crucial link between the distant and nearby Universe.

By studying gravitationally lensed EELGs, we can probe the physics of galaxy formation in an environment that closely mimics the conditions of Cosmic Dawn, but with a clarity and detail that is currently unachievable at the highest redshifts. This synergistic approach does not replace the direct observations of JWST; rather, it enriches them, providing a low-redshift, high-fidelity roadmap for interpreting the signals from the dawn of cosmic history. The ongoing and future surveys with instruments like MUSE will continue to build the samples of lensed EELGs needed to fully exploit this powerful technique.

\bibliographystyle{mnras}
\bibliography{references}

\bsp
\label{lastpage}
\end{document}
